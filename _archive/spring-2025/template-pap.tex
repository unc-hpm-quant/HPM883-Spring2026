% Options for packages loaded elsewhere
\PassOptionsToPackage{unicode}{hyperref}
\PassOptionsToPackage{hyphens}{url}
\PassOptionsToPackage{dvipsnames,svgnames,x11names}{xcolor}
%
\documentclass[
  letterpaper,
  DIV=11,
  numbers=noendperiod]{scrartcl}

\usepackage{amsmath,amssymb}
\usepackage{iftex}
\ifPDFTeX
  \usepackage[T1]{fontenc}
  \usepackage[utf8]{inputenc}
  \usepackage{textcomp} % provide euro and other symbols
\else % if luatex or xetex
  \usepackage{unicode-math}
  \defaultfontfeatures{Scale=MatchLowercase}
  \defaultfontfeatures[\rmfamily]{Ligatures=TeX,Scale=1}
\fi
\usepackage{lmodern}
\ifPDFTeX\else  
    % xetex/luatex font selection
\fi
% Use upquote if available, for straight quotes in verbatim environments
\IfFileExists{upquote.sty}{\usepackage{upquote}}{}
\IfFileExists{microtype.sty}{% use microtype if available
  \usepackage[]{microtype}
  \UseMicrotypeSet[protrusion]{basicmath} % disable protrusion for tt fonts
}{}
\makeatletter
\@ifundefined{KOMAClassName}{% if non-KOMA class
  \IfFileExists{parskip.sty}{%
    \usepackage{parskip}
  }{% else
    \setlength{\parindent}{0pt}
    \setlength{\parskip}{6pt plus 2pt minus 1pt}}
}{% if KOMA class
  \KOMAoptions{parskip=half}}
\makeatother
\usepackage{xcolor}
\setlength{\emergencystretch}{3em} % prevent overfull lines
\setcounter{secnumdepth}{-\maxdimen} % remove section numbering
% Make \paragraph and \subparagraph free-standing
\makeatletter
\ifx\paragraph\undefined\else
  \let\oldparagraph\paragraph
  \renewcommand{\paragraph}{
    \@ifstar
      \xxxParagraphStar
      \xxxParagraphNoStar
  }
  \newcommand{\xxxParagraphStar}[1]{\oldparagraph*{#1}\mbox{}}
  \newcommand{\xxxParagraphNoStar}[1]{\oldparagraph{#1}\mbox{}}
\fi
\ifx\subparagraph\undefined\else
  \let\oldsubparagraph\subparagraph
  \renewcommand{\subparagraph}{
    \@ifstar
      \xxxSubParagraphStar
      \xxxSubParagraphNoStar
  }
  \newcommand{\xxxSubParagraphStar}[1]{\oldsubparagraph*{#1}\mbox{}}
  \newcommand{\xxxSubParagraphNoStar}[1]{\oldsubparagraph{#1}\mbox{}}
\fi
\makeatother


\providecommand{\tightlist}{%
  \setlength{\itemsep}{0pt}\setlength{\parskip}{0pt}}\usepackage{longtable,booktabs,array}
\usepackage{calc} % for calculating minipage widths
% Correct order of tables after \paragraph or \subparagraph
\usepackage{etoolbox}
\makeatletter
\patchcmd\longtable{\par}{\if@noskipsec\mbox{}\fi\par}{}{}
\makeatother
% Allow footnotes in longtable head/foot
\IfFileExists{footnotehyper.sty}{\usepackage{footnotehyper}}{\usepackage{footnote}}
\makesavenoteenv{longtable}
\usepackage{graphicx}
\makeatletter
\newsavebox\pandoc@box
\newcommand*\pandocbounded[1]{% scales image to fit in text height/width
  \sbox\pandoc@box{#1}%
  \Gscale@div\@tempa{\textheight}{\dimexpr\ht\pandoc@box+\dp\pandoc@box\relax}%
  \Gscale@div\@tempb{\linewidth}{\wd\pandoc@box}%
  \ifdim\@tempb\p@<\@tempa\p@\let\@tempa\@tempb\fi% select the smaller of both
  \ifdim\@tempa\p@<\p@\scalebox{\@tempa}{\usebox\pandoc@box}%
  \else\usebox{\pandoc@box}%
  \fi%
}
% Set default figure placement to htbp
\def\fps@figure{htbp}
\makeatother

\KOMAoption{captions}{tableheading}
\makeatletter
\@ifpackageloaded{caption}{}{\usepackage{caption}}
\AtBeginDocument{%
\ifdefined\contentsname
  \renewcommand*\contentsname{Table of contents}
\else
  \newcommand\contentsname{Table of contents}
\fi
\ifdefined\listfigurename
  \renewcommand*\listfigurename{List of Figures}
\else
  \newcommand\listfigurename{List of Figures}
\fi
\ifdefined\listtablename
  \renewcommand*\listtablename{List of Tables}
\else
  \newcommand\listtablename{List of Tables}
\fi
\ifdefined\figurename
  \renewcommand*\figurename{Figure}
\else
  \newcommand\figurename{Figure}
\fi
\ifdefined\tablename
  \renewcommand*\tablename{Table}
\else
  \newcommand\tablename{Table}
\fi
}
\@ifpackageloaded{float}{}{\usepackage{float}}
\floatstyle{ruled}
\@ifundefined{c@chapter}{\newfloat{codelisting}{h}{lop}}{\newfloat{codelisting}{h}{lop}[chapter]}
\floatname{codelisting}{Listing}
\newcommand*\listoflistings{\listof{codelisting}{List of Listings}}
\makeatother
\makeatletter
\makeatother
\makeatletter
\@ifpackageloaded{caption}{}{\usepackage{caption}}
\@ifpackageloaded{subcaption}{}{\usepackage{subcaption}}
\makeatother

\usepackage{bookmark}

\IfFileExists{xurl.sty}{\usepackage{xurl}}{} % add URL line breaks if available
\urlstyle{same} % disable monospaced font for URLs
\hypersetup{
  pdftitle={Pre-Analysis Plan Template},
  pdfauthor={Your Name},
  colorlinks=true,
  linkcolor={blue},
  filecolor={Maroon},
  citecolor={Blue},
  urlcolor={Blue},
  pdfcreator={LaTeX via pandoc}}


\title{Pre-Analysis Plan Template}
\author{Your Name}
\date{Invalid Date}

\begin{document}
\maketitle

\renewcommand*\contentsname{Table of contents}
{
\hypersetup{linkcolor=}
\setcounter{tocdepth}{3}
\tableofcontents
}

This is a template for a Pre-Analysis Plan (PAP), roughly following
guidelines from the American Economic Association (AEA) and the World
Bank's Development Impact Evaluation (DIME) group.

\subsection{Helpful Resources:}\label{helpful-resources}

\begin{itemize}
\item
  For guidance on pre-analysis plans, refer to the World Bank's DIME
  Wiki:
  \href{https://dimewiki.worldbank.org/Pre-Analysis_Plan}{Pre-Analysis
  Plan - DIME Wiki}
\item
  For examples of pre-analysis plans, explore the AEA's RCT Registry:
  \href{https://www.socialscienceregistry.org/}{AEA RCT Registry}
\end{itemize}

\section{Title}\label{title}

\textbf{{[}Descriptive Title of the Study{]}}

\section{Introduction}\label{introduction}

{[}Provide a brief overview of the study, including the research
question, context, and objectives.{]}

\section{Hypotheses}\label{hypotheses}

\begin{itemize}
\tightlist
\item
  \textbf{Primary Hypothesis:} {[}State the main hypothesis to be
  tested.{]}
\item
  \textbf{Secondary Hypotheses:} {[}List any additional hypotheses.{]}
\end{itemize}

\section{Experimental Design}\label{experimental-design}

\begin{itemize}
\tightlist
\item
  \textbf{Intervention Details:} {[}Describe the intervention or
  treatment.{]}
\item
  \textbf{Control Group:} {[}Describe the control condition.{]}
\item
  \textbf{Randomization:} {[}Explain the randomization process,
  including stratification or blocking if applicable.{]}
\end{itemize}

\section{Outcome Measures}\label{outcome-measures}

\begin{itemize}
\tightlist
\item
  \textbf{Primary Outcomes:} {[}Define the main outcomes to be
  measured.{]}
\item
  \textbf{Secondary Outcomes:} {[}Define additional outcomes of
  interest.{]}
\end{itemize}

\section{Data Collection}\label{data-collection}

\begin{itemize}
\tightlist
\item
  \textbf{Timing:} {[}Specify when data will be collected (e.g.,
  baseline, endline).{]}
\item
  \textbf{Methods:} {[}Describe data collection methods and
  instruments.{]}
\end{itemize}

\section{Sample Size and Power
Calculations}\label{sample-size-and-power-calculations}

\begin{itemize}
\tightlist
\item
  \textbf{Sample Size:} {[}State the planned sample size.{]}
\item
  \textbf{Power Calculations:} {[}Provide details of power calculations,
  including assumptions and effect sizes.{]}
\end{itemize}

\section{Empirical Analysis}\label{empirical-analysis}

\begin{itemize}
\tightlist
\item
  \textbf{Estimation Strategy:} {[}Describe the statistical models and
  methods to be used.{]}
\item
  \textbf{Handling of Missing Data:} {[}Explain how missing data will be
  addressed.{]}
\item
  \textbf{Multiple Hypothesis Testing:} {[}Discuss adjustments for
  multiple comparisons, if applicable.{]}
\end{itemize}

\section{Heterogeneous Treatment
Effects}\label{heterogeneous-treatment-effects}

{[}Specify any subgroup analyses or interactions to assess heterogeneous
effects.{]}

\section{Robustness Checks}\label{robustness-checks}

{[}Outline any additional analyses to test the robustness of the
results.{]}

\section{Data Management}\label{data-management}

\begin{itemize}
\tightlist
\item
  \textbf{Data Cleaning:} {[}Describe procedures for data cleaning and
  validation.{]}
\item
  \textbf{Replication:} {[}Plan for sharing replication files and
  data.{]}
\end{itemize}

\section{Ethical Considerations}\label{ethical-considerations}

{[}Discuss ethical approvals, consent procedures, and any potential
risks to participants.{]}

\section{References}\label{references}

{[}Include all references cited in the plan.{]}




\end{document}
